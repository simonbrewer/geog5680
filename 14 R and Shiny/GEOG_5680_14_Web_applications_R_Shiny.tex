\documentclass[aspectratio=169]{beamer}\usepackage[]{graphicx}\usepackage[]{color}
% maxwidth is the original width if it is less than linewidth
% otherwise use linewidth (to make sure the graphics do not exceed the margin)
\makeatletter
\def\maxwidth{ %
  \ifdim\Gin@nat@width>\linewidth
    \linewidth
  \else
    \Gin@nat@width
  \fi
}
\makeatother

\definecolor{fgcolor}{rgb}{0.345, 0.345, 0.345}
\newcommand{\hlnum}[1]{\textcolor[rgb]{0.686,0.059,0.569}{#1}}%
\newcommand{\hlstr}[1]{\textcolor[rgb]{0.192,0.494,0.8}{#1}}%
\newcommand{\hlcom}[1]{\textcolor[rgb]{0.678,0.584,0.686}{\textit{#1}}}%
\newcommand{\hlopt}[1]{\textcolor[rgb]{0,0,0}{#1}}%
\newcommand{\hlstd}[1]{\textcolor[rgb]{0.345,0.345,0.345}{#1}}%
\newcommand{\hlkwa}[1]{\textcolor[rgb]{0.161,0.373,0.58}{\textbf{#1}}}%
\newcommand{\hlkwb}[1]{\textcolor[rgb]{0.69,0.353,0.396}{#1}}%
\newcommand{\hlkwc}[1]{\textcolor[rgb]{0.333,0.667,0.333}{#1}}%
\newcommand{\hlkwd}[1]{\textcolor[rgb]{0.737,0.353,0.396}{\textbf{#1}}}%
\let\hlipl\hlkwb

\usepackage{framed}
\makeatletter
\newenvironment{kframe}{%
 \def\at@end@of@kframe{}%
 \ifinner\ifhmode%
  \def\at@end@of@kframe{\end{minipage}}%
  \begin{minipage}{\columnwidth}%
 \fi\fi%
 \def\FrameCommand##1{\hskip\@totalleftmargin \hskip-\fboxsep
 \colorbox{shadecolor}{##1}\hskip-\fboxsep
     % There is no \\@totalrightmargin, so:
     \hskip-\linewidth \hskip-\@totalleftmargin \hskip\columnwidth}%
 \MakeFramed {\advance\hsize-\width
   \@totalleftmargin\z@ \linewidth\hsize
   \@setminipage}}%
 {\par\unskip\endMakeFramed%
 \at@end@of@kframe}
\makeatother

\definecolor{shadecolor}{rgb}{.97, .97, .97}
\definecolor{messagecolor}{rgb}{0, 0, 0}
\definecolor{warningcolor}{rgb}{1, 0, 1}
\definecolor{errorcolor}{rgb}{1, 0, 0}
\newenvironment{knitrout}{}{} % an empty environment to be redefined in TeX

\usepackage{alltt}
\usepackage{multirow}
%\usecolortheme{beaver}
%\usecolortheme[RGB={129,3,3}]{structure}
\usetheme{CambridgeUS}
\usecolortheme{seahorse}

% Standard header (will need to change date!)
\title[GEOG 5680 Summer '20]{GEOG 5680\\Introduction to R}
\subtitle[Intro]{14: Web applications with R and \textbf{Shiny}}
\author[S. Brewer]{Simon Brewer}
\institute[Univ. Utah]{
  Geography Department\\
  University of Utah\\
  Salt Lake City, Utah 84112\\[1ex]
  \texttt{simon.brewer@geog.utah.edu}
}
\date[May 07, 2020]{May 07, 2020}
\IfFileExists{upquote.sty}{\usepackage{upquote}}{}
\begin{document}


%--- the titlepage frame -------------------------%
\begin{frame}
  \titlepage
\end{frame}

% \section{Objectives}
% %--- Slide 3 ----------------%
% \begin{frame}{Objectives}
% \begin{itemize}
%   \item Introduce Shiny
%   \item Simple example
%   \item Widgets
%   \item Deployment options
% \end{itemize}
% \end{frame}

\section{Shiny}
%--- Slide ----------------%
\begin{frame}[fragile]{What is Shiny?}

\begin{columns}
	\begin{column}{0.5\textwidth}
  \begin{itemize}
    	\item Open source web application for R (from RStudio)
    	\item Turn analysis into interactive web app
    	\item No real knowledge of HTML, CSS or javascript needed
    	\item Examples: https://shiny.rstudio.com/gallery/
  \end{itemize}
	\end{column}
	\begin{column}{0.5\textwidth}
  \begin{center}
      \includegraphics[width=0.65\textwidth]{./images/shiny2.jpg}
  \end{center}
	\end{column}
\end{columns}
\end{frame}

%--- Slide ----------------%
\begin{frame}[fragile]{Shiny applications}
Can use single script (app.R), but for more complex applications, easier to work with two scripts
\begin{columns}
	\begin{column}{0.5\textwidth}
    \begin{itemize}
      \item \emph{ui}: user interface
    	\item Controls user interface
    	\item Layout, appearance
    	\item Widgets for input that changes behavior of app
    	\item Displays output from \emph{server}
    \end{itemize}
	\end{column}
	\begin{column}{0.5\textwidth}
  \begin{itemize}
      \item \emph{server}: server
    	\item Imports parameters from \emph{ui}
    	\item Uses these to run analysis and produces output
    	%\item Most of your R code will go here
    	\item Defines two objects (\texttt{input} and \texttt{output}) which are used to transfer parameters between UI and server
  \end{itemize}
	\end{column}
\end{columns}
\end{frame}

%--- Slide ----------------%
\begin{frame}[fragile]{Shiny applications}
Simple example of *ui.R*:
\begin{knitrout}\scriptsize
\definecolor{shadecolor}{rgb}{0.969, 0.969, 0.969}\color{fgcolor}\begin{kframe}
\begin{alltt}
\hlkwd{library}\hlstd{(shiny)}
\hlkwd{shinyUI}\hlstd{(}\hlkwd{fluidPage}\hlstd{(}
  \hlkwd{titlePanel}\hlstd{(}\hlstr{"Hello Shiny!"}\hlstd{),}
  \hlkwd{sidebarLayout}\hlstd{(}
    \hlkwd{sidebarPanel}\hlstd{(}
      \hlkwd{sliderInput}\hlstd{(}\hlstr{"bins"}\hlstd{,}
                  \hlstr{"Number of bins:"}\hlstd{,}
                  \hlkwc{min} \hlstd{=} \hlnum{1}\hlstd{,}
                  \hlkwc{max} \hlstd{=} \hlnum{50}\hlstd{,}
                  \hlkwc{value} \hlstd{=} \hlnum{30}\hlstd{)}
    \hlstd{),}
    \hlkwd{mainPanel}\hlstd{(}
      \hlkwd{plotOutput}\hlstd{(}\hlstr{"distPlot"}\hlstd{)}
    \hlstd{)}
  \hlstd{)}
\hlstd{))}
\end{alltt}
\end{kframe}
\end{knitrout}
\end{frame}

%--- Slide ----------------%
\begin{frame}[fragile]{Shiny applications}
Simple example of *server.R*:
\begin{knitrout}\scriptsize
\definecolor{shadecolor}{rgb}{0.969, 0.969, 0.969}\color{fgcolor}\begin{kframe}
\begin{alltt}
\hlkwd{library}\hlstd{(shiny)}
\hlkwd{shinyServer}\hlstd{(}\hlkwa{function}\hlstd{(}\hlkwc{input}\hlstd{,} \hlkwc{output}\hlstd{) \{}
  \hlstd{output}\hlopt{$}\hlstd{distPlot} \hlkwb{<-} \hlkwd{renderPlot}\hlstd{(\{}
    \hlstd{x}    \hlkwb{<-} \hlstd{faithful[,} \hlnum{2}\hlstd{]}  \hlcom{# Old Faithful Geyser data}
    \hlstd{bins} \hlkwb{<-} \hlkwd{seq}\hlstd{(}\hlkwd{min}\hlstd{(x),} \hlkwd{max}\hlstd{(x),} \hlkwc{length.out} \hlstd{= input}\hlopt{$}\hlstd{bins} \hlopt{+} \hlnum{1}\hlstd{)}
    \hlkwd{hist}\hlstd{(x,} \hlkwc{breaks} \hlstd{= bins,} \hlkwc{col} \hlstd{=} \hlstr{'darkgray'}\hlstd{,} \hlkwc{border} \hlstd{=} \hlstr{'white'}\hlstd{)}
  \hlstd{\})}
\hlstd{\})}
\end{alltt}
\end{kframe}
\end{knitrout}
\end{frame}

%--- Slide ----------------%
\begin{frame}[fragile]{Shiny applications}
Example output
\end{frame}

%--- Slide ----------------%
\begin{frame}[fragile]{Shiny applications}
  \begin{itemize}
    	\item Easy way to make simple web applications for analysis
    	\item Integrates with RMarkdown
    	\item Can use all standard R add-ons, including \textbf{ggplot2}, etc
    	\item Can be deployed using personal Shiny server or through RStudio's site
  \end{itemize}
\end{frame}

% %--- Slide ----------------%
% \begin{frame}{Next Class}
% \begin{itemize}
%   \item Lab: Web applications
%   \item 0503: Interactive plotting
% \end{itemize}
% \end{frame}

\end{document}
